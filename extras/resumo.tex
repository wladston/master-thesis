O trabalho em equipe está presente em várias atividades importantes na
sociedade, como por exemplo na produção de filmes e shows, de trabalhos
científicos, nos esportes e em escritórios. Na maioria dos casos, os agentes
que trabalham em uma mesma indústria se organizam em uma rede social, e times
são formados não de forma aleatória, mas no contexto dessa rede. Em trabalhos
colaborativos, é necessário que os agentes escolham suas equipes, idealmente
seguindo uma estratégia que maximize a probabilidade de formação de um time de
sucesso. No presente trabalho, considerando a tarefa de prever o sucesso de um
time inserido em uma rede social, avaliamos o poder preditivo de
características puramente sociais de times em comparação a outros fatores
não-sociais sabidamente correlacionados com o sucesso de equipes. Ao contrário
de outros trabalhos propostos para prever o sucesso de times ou avaliar a
influência de fatores sociais no sucesso de equipes, nossa abordagem considera
múltiplos fatores sociais, um grande número de times e uma vasta rede social.
Nossa abordagem consiste em utilizar dados históricos de co-produção de filmes
para montar um grafo representando a rede social de agentes produtores de
filmes, e extrair características sociais e não sociais dos times de produção
de cada filme utilizando esse grafo, para em seguida avaliar o poder preditivo
dessas características em relação ao sucesso dos filmes. Apresentamos uma
caracterização da rede de produção de cinema ao longo das décadas e uma
avaliação do poder preditivo dos fatores topológicos estudados, juntamente com
sugestões de como esses resultados podem ser utilizados na composição de novos
times. Nossos resultados mostram que algumas características topológicos de
times ajudam a tarefa de previsão de sucesso, complementando as demais
características que já se sabiam ser relacionadas ao sucesso de times.

\keywords{Redes Sociais, Formação de Equipes, Análise de Regressão, Previsão de
Sucesso}
