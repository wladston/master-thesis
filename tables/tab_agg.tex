\begin{table}[t]
\caption{\label{tab:agg} Summary of aggregation techniques used to generate features.}
\centering
\begin{tabular}{@{}p{3.6cm}@{} p{8cm}@{}}
\toprule
\textbf{Aggregator} & \textbf{Description}\\
\midrule
\parbox[t]{2.0cm}{Arithmetic\\mean} & Indicates the central tendency or typical value of a set of numbers by using the sum of their values.\\
\midrule
\parbox[t]{2.0cm}{Harmonic\\mean} &  Indicates the central tendency of a set of rates.\\
\midrule
Median & Number separating the higher half of  the team producers metrics from the lower half.  \\
\midrule
\parbox[t]{2.0cm}{Minimum, Maximum} &  Lowest and highest values of the team producers metrics.\\
\midrule
\parbox[t]{2.0cm}{Standard\\Deviation} & Amount of variation or dispersion of the team producers metrics.\\
\midrule
\parbox[t]{2.0cm}{Node\\Contraction} & Combines multiple nodes in a graph into one, so aggregate information can be retrieved from the super node.\\
\bottomrule
\end{tabular}
\end{table}
