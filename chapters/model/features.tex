\section{Defining Features for Movies}
\label{sec:features}

This section presents the features extracted from each movie, which can be
divided into two main groups: features related to characteristics from the
movie producing team and features related to the movie itself.
Table~\ref{tab:features} presents an overview of all features.

\begin{table}[h]
\caption{\label{tab:features}All 121 features considered in the experiment.}

\begin{subtable}[b]{\textwidth}
\caption{\textbf{Features obtained via aggregation}}
\centering
\begin{tabular}{@{}llcc@{}} \toprule
Feature Type & Metric Name & \xcell{Number of\\Aggregations} & \xcell{ Number of\\ Features} \\ \midrule
\multirow{10}{*}{\xcell{Ego\\Metrics}} & Closeness  & 7 & \multirow{10}{*}{70}    \\
        & Betweenness  &  7  &         \\
        & Clustering   & 7   &         \\
        & Square clustering & 7 &      \\
        & Degree                     & 7                          &                         \\
                        & Net.\ constraint           & 7                          &                         \\
                        & Prev.\ rating              & 7                          &                         \\
                        & Prev.\ gross               & 7                          &                         \\
                        & Prev.\ votes               & 7                          &                         \\
                        & Prev.\ experience          & 7                          &                         \\ \midrule
\multirow{3}{*}{\xcell{Pairwise\\Metrics}} & Past experience            & 6                          & \multirow{3}{*}{18}     \\
                        & Shared friends             & 6                          &                         \\
                        & Neighbour overlap          & 6                          &                         \\ \bottomrule
\end{tabular}
\end{subtable}

\vspace{6pt}

\begin{subtable}[b]{\textwidth}
\caption{\textbf{Features obtained via One-Hot Encoding}}
\centering
\begin{tabular}{@{}llcc@{}} \toprule
Feature Type             & Characteristic                & \xcell{Number of\\Groups}   & \xcell{Number of\\Features}          \\ \midrule
\multirow{2}{*}{\xcell{Movie\\Characteristics}}  & Genres                     & 21                         & \multirow{2}{*}{27}     \\
                        & Continents                 & 6                          &                         \\ \bottomrule
\end{tabular}
\end{subtable}

\vspace{6pt}

\begin{subtable}[b]{\textwidth}
\caption{\textbf{Simple Features}}
\centering
\begin{tabular}{@{}llcc@{}} \toprule
Feature Type             & \multicolumn{2}{l}{Feature Name}                          & \xcell{Number of\\Features}           \\ \midrule
\multirow{3}{*}{\xcell{Global\\Metrics}} & \multicolumn{2}{l}{Global clustering coefficient}                    & \multirow{3}{*}{3}      \\
                        & \multicolumn{2}{l}{Average path length}                          &                         \\
                        & \multicolumn{2}{l}{Small world coefficient}                          &                         \\ \midrule
\multirow{3}{*}{\xcell{Movie\\Characteristics}}   & \multicolumn{2}{l}{Runtime in minutes}    &    \multirow{3}{*}{3}   \\
    & \multicolumn{2}{l}{Production team size}  &            \\
    & \multicolumn{2}{l}{Production budget (normalized)}  &    \\
\bottomrule
\end{tabular}
\end{subtable}
\end{table}


\subsection{From Movie Characteristics}
We include 32 features that are unrelated to social characteristics from the
movie production team composition: the movie's genres, production countries,
runtime length, and the global state of the interaction graph model
(Section~\ref{sec:fundamentals:social}) at the time of the movie's release.
These characteristics have also been previously explored to predict movie
success~\citep{Ghiassi2015} and influence~\citep{wasserman2015cross}.  Next, we
describe the methodology for extracting these features.

Movie genre is categorical data: the IMDb dataset may assign one or more genres
to movies. Therefore, we used 21 binary features to represent this categorical
data, each one representing whether the movie belongs to a specific genre.

The IMDb dataset also provides a list of countries associated with the
production of a movie. Since there are about 200 different countries, it is not
efficient to encode country production into features directly. Therefore, the
UN list for countries and continents was used to transform the movie's list of
related countries into this list of related ``geographic regions'': Africa,
Asia, Europe, Latin America, North America and Oceania. Again, binary features
were used, each encoding whether any country of the given continent took part
in producing the movie. We note that movies can be jointly produced by teams in
different countries, or by teams residing in a single country.

Runtime length and production team size are directly represented as integer
features. Runtimes range from 40 minutes to 450
minutes\footnote{``S\'at\'antang\'o'' by B\'ela Tarr (1994) has a runtime of
432 minutes.}. Production team sizes range from 2 to 19\footnote{Production
team of ``Knuckleball!'' (2012).}.

Finally, we calculate topological characteristics concerning the state of the
whole network (following~\cite{uzzi2005collaboration}): its clustering
coefficient, the average path length, and its small world coefficient.

\subsection{From Movie Production Teams}
Here, we present features derived from the set of social characteristics and
past history from movie producing teams, which are based on the aspects
presented in Table~\ref{tab:team_metrics}. In addition to each producer's track
record, the table contains well known topological metrics regarding social
network analysis, which capture different social aspects of nodes in a social
network graph.

We consider six topological ego-metrics (closeness, betweenness, clustering,
square clustering, degree, and network constraint) and two pairwise-metrics
(shared friends and neighbor overlap). These metrics are defined for singular
nodes or pairs of nodes, and there is no standard methodology for extracting
such topological information from sets of two or more nodes that constitute a
team. To address this problem, we aggregate topological information from nodes
in teams in two different ways: (\textit{i}) we aggregate values from metrics
calculated from single nodes (or pairs of nodes) in the team, and (\textit{ii})
we temporarily contract all nodes from the team into a single node (using node
contraction, Chapter~\ref{sec:fundamentals:social}) and calculate metrics in
the contracted node\footnote{Node contraction cannot be used with pair-wise
metrics, since it aggregates all information into a single node}.

In the first approach, aggregating the different metric values in a single way
(e.g., considering only the simple mean of the values) is prone to information
loss, since many distinct distributions can have the same mean value.
Therefore, in addition to the mean, five other statistical aggregation
functions are used. This way, more information about the distribution of the
values is preserved. The seven methods for statistical and non-statistical
aggregation are briefly presented in Table~\ref{tab:agg}.

Moreover, we consider three global metrics to get the network state at time of
movie release: the global clustering coefficient, the average path length and
the small world coefficient. Hence, we consider 57 topological features (six
ego-metrics aggregated in seven ways, two pairwise metrics aggregated in six
ways, plus three global ones).

\begin{table}[H]
\centering
\caption{\label{tab:team_metrics}Summary of all metrics used for team characterization.}

\begin{subtable}[b]{\textwidth}
\caption{\textbf{Ego metrics}}
\centering
\begin{tabular}{@{}p{3cm}@{} p{12.0cm}@{}}
%\toprule
%\multicolumn{2}{@{}l}{\textbf{Ego metrics}}\\
\toprule
Closeness & Captures how close a producer $p_k$ is from all other producers reachable from it in $\mathbb{G}_\mathbb{P}$.\\
\midrule
Betweenness & Fraction of all shortest paths, computed using breadth-first search, connecting pairs of producers that pass through a particular producer $p_k$.\\
\midrule
Clustering & Fraction of pairs of producer $p_k$ collaborators who have also collaborated with one another.\\
\midrule
\parbox[t]{2.5cm}{Square \\ clustering} & Fraction of possible cycles of size 4 that exist at the node~\citep{lind2005cycles}.\\
\midrule
\parbox[t]{2.5cm}{Network \\ constrain} & Index measuring the extent of the bridiging of the node (whether the node connects different clusters~\citep{burt2005brokerage}).\\
\midrule
Degree & Total number of partners of a producer $p_k$.\\
\midrule
\parbox[t]{2.5cm}{Past \\ experience} & Number of prior movies produced by the node.\\
\midrule
\parbox[t]{2.5cm}{Previous \\ success} & Mean success metrics for prior movies produced by the node, defined for all three success parameters extracted from movies.\\
\bottomrule
\end{tabular}
\end{subtable}

\vspace{6pt}

\begin{subtable}[b]{\textwidth}
\caption{\textbf{Pair-wise metrics}}
\centering
%\addlinespace[2.5ex]
%\multicolumn{2}{@{}l}{\textbf{Pair-wise metrics}}   \\
\begin{tabular}{@{}p{3cm}@{} p{12.0cm}@{}}
\toprule
\parbox[t]{2.5cm}{Shared \\ friends} & Number of nodes connected to both nodes in the pair.\\
\midrule
\parbox[t]{2.5cm}{Neighbour \\ overlap} & Rate of shared friends and total nodes connected to the pair. \\
\midrule
\parbox[t]{2.5cm}{Shared \\ experience} & Number of prior movies jointly produced by the pair of nodes. \\
\bottomrule
\end{tabular}
\end{subtable}

\vspace{6pt}

\begin{subtable}[b]{\textwidth}
\caption{\textbf{Global team and graph metrics}}
\centering
\begin{tabular}{@{}p{3cm}@{} p{12.0cm}@{}}

%\addlinespace[2.5ex]
%\multicolumn{2}{@{}l}{\textbf{Global team and graph metrics}} \\
\toprule
\parbox[t]{2.5cm}{Global \\ clustering} & Measure of degree to which the producers in $\mathbb{G}_\mathbb{P}$ tend to cluster together.\\
\midrule
\parbox[t]{2.5cm}{Average\\ shortest path} & Average across all values of shortest path for all pairs of producers in $\mathbb{G}_\mathbb{P}$.\\
\midrule
\parbox[t]{2.5cm}{Team Size} & The total size of full production team.\\
\bottomrule
\end{tabular}
\end{subtable}

\end{table}


\begin{table}[t]
\caption{\label{tab:agg} Summary of aggregation techniques used to generate features.}
\centering
\begin{tabular}{@{}p{3.6cm}@{} p{8cm}@{}}
\toprule
\textbf{Aggregator} & \textbf{Description}\\
\midrule
\parbox[t]{2.0cm}{Arithmetic\\mean} & Indicates the central tendency or typical value of a set of numbers by using the sum of their values.\\
\midrule
\parbox[t]{2.0cm}{Harmonic\\mean} &  Indicates the central tendency of a set of rates.\\
\midrule
Median & Number separating the higher half of  the team producers metrics from the lower half.  \\
\midrule
\parbox[t]{2.0cm}{Minimum, Maximum} &  Lowest and highest values of the team producers metrics.\\
\midrule
\parbox[t]{2.0cm}{Standard\\Deviation} & Amount of variation or dispersion of the team producers metrics.\\
\midrule
\parbox[t]{2.0cm}{Node\\Contraction} & Combines multiple nodes in a graph into one, so aggregate information can be retrieved from the super node.\\
\bottomrule
\end{tabular}
\end{table}


The node's track record is given by four measures related to its past success
and experience (average previous rating, gross, votes and number of previously
produced movies) and one pairwise metric (number of movies previously jointly
produced by the pair). The former may be grouped by using the seven
aforementioned aggregation techniques, yielding 28 features. The latter may be
grouped by the six mathematical aggregation functions, generating six more
features. At the end, this part considers 34 features.

With these characteristics, we capture a complete composition of the team into
numerical and analytical features, such as: 

\begin{itemize}
\item whether they are composed of amateurs, experienced producers, or mixed,
by looking at the teams' past experience metric, aggregate in terms of
mean, minimum, maximum and standard deviation;
\item whether there is a producer who has had a spectacular track record in the
team, by looking at the teams' previous success metric, aggregated in terms of
mean and harmonic mean;
\item whether the team has a central position in the whole network, by looking
at team's betweenness metric, aggregated using the team's \textit{supernode};
\item whether they are strongly or weakly connected to other producers in the
industry, by looking at the degree and network constraint metrics,
aggregated by team's \textit{supernode}.
\end{itemize}
