\chapter{Conclusion and Future Work}
\label{chapter:conclusion}
In the context of forming high-achieving collaboration teams and predicting
whether a collaboration team will produce successful work, we present a
predictive analysis of movie producing team's success. This analysis employs
features and characteristics from previous work, as well as novel topological
characteristics derived from the agent's social ties. We found certain patterns
in topological organization (such as low network constraint and high closeness)
of teams are associated with success. This has profound implications in team
formation research. This work is a first glance at how topology of teams
affects movie performance. However, we hope this claim may be generalized
beyond movie production teams.

Such association between network topology of collaboration teams and success
could be used in many practical applications. For instance, it could be used to
guide strategies for helping managers responsible for team formation to
assemble teams with higher performance. Hence, they could optimize their
revenue capacity and social impact. The results could also be used to develop
tools that automatically analyze online enterprise social networks. This could
be achieved by assessing how far teams on the network are from the best
performing configurations. Future algorithms could suggest changes in the
network in order to achieve such configurations. Nonetheless, one fundamental
challenge is to acquire the dataset to support a predictive model.

This study revealed how topological and non-topological characteristics of
movie production teams are related to movie success. There has been prior work
on the impact of topological characteristics on movie performance and success.
In such a context, this work is novel on aggregating topological features from
teams. Moreover, this was performed on a very large dataset (whereas previous
work considers just a small fraction of movies). However, we also show a larger
and more heterogeneous dataset can decrease prediction accuracy. This suggests
that movie success prediction may require more sophisticated models that
further consider temporal phenomena and topological aspects.

Moreover, we found the topological organization of global movie production
teams is steadily evolving throughout the years. Our study confirms the ``rich
get richer'' phenomenon with movies: teams with a better success past are more
likely to produce successful movies. However, our study also confirms the
importance of including fresh producers into teams, who act as bridges in the
social network. 

In the future, we plan to improve the network model by developing more
effective ways to work with the network graph. Doing so could allow considering
the movie's full cast and crew for example. Another interesting direction would
be to apply this methodology for analyzing other types of collaborative
networks (such as the one formed by research paper publishers or by corporative
knowledge workers performing teamwork). This would allow confirming that the
social structure of these networks change profoundly with time, and that social
metrics from their collaboration effort can help explaining team success.

The prediction model could also be improved in many ways, such as by using more
advanced means for feature selection. Also, more complex machine learning
tools, such as artificial neural networks, which are able to detect non-linear
relationships, could be used to possibly further improve the prediction
accuracy. The model could be extended to predict other relevant factors such as
forecasting which movies are going to be Oscar winners. The model could also be
extended by considering new features, such as the presence of famous actors.

Finally, the initial results of this dissertation were published
in~\cite{laweb}. The main results were published in~\cite{sac}.
