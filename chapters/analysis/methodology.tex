\section{Methodology}
\label{sec:method}
Choosing a single split of train/test sets might produce poor experimental
results, because the train and test sets could be selected in a biased
way~\citep{mohri2012foundations}. To avoid this problem, we perform
\textit{K-fold Cross Validation} with $K = 5$.  The selection of $K$ was made
such that it effectively controls the bias while it generates large folds that
are more heterogeneous. For each cross validation, the movies are randomly
split into five distinct groups, and five train/test cycles are performed. Each
time, one of the groups is chosen to be the test group and the other groups are
used for training the regression model. In each train/test cycle, the
coefficient of determination\footnote{Throughout this chapter, the coefficient
of determination (denoted as \textbf{$R^2$}) is used to assess how well the
different regression models are predicting movie success according to their
given inputs.} is computed by evaluating the model using the respective test
set. This gives five $R^2$ values for each run. We consider the mean of these
values as the result of the cross-validation execution.

For each evaluation, it is important to ensure that the special five folds were
not randomized in a biased way. Hence, when evaluating a model, we take results
from 30 different \textit{5-fold cross validation} executions. We then
calculate the mean and confidence intervals ($\alpha$ = 95\%) from such runs as
the model's final score.
 
If the distribution of votes and gross are highly skewed, forming a power law 
(few movies with most votes and gross, see Figure~\ref{fig:hist_success}), 
there is a high probability that either no or an insufficient amount of 
high-performing movies will be present in several randomized folds. Such skewness 
might favor the formation of biased train/test sets. This could impair the model's 
ability to recognize higher performing movies and thus compromise its accuracy.

To account for this phenomenon, we attempted under-sampling the number of
movies with lower levels of success in the training sets. However, following
experimental evaluation showed this does \textit{not} improve regression
accuracy.  Hence, we choose to generate \textit{balanced} train/test folds.
Using \textit{balanced} folds, train and test sets of movies are grouped in a
way that ensures groups have similar distributions of movies among each of the
three performance groups ($G_1$, $G_2$ and $G_3$ as defined in
Table~\ref{tab:success_groups}). Moreover, all train/test folds are still
generated in a random fashion.

Finally, Algorithm~\ref{algo:predictor} provides an overview of all major steps
and processes involved in the experiments. Briefly, it shows all steps starting
from the raw data all the way to a trained movie success prediction model and
its performance.

\begin{algorithm}[tb]
 \SetKwInOut{Input}{input}\SetKwInOut{Output}{output}
 \Input{Raw IMDb Movie Data}
 \Output{Movie success prediction model}
    $\mathbb{G} \leftarrow$ [~] \tcc*[r]{producers' graph}
    movie\_map $\leftarrow$ [~] \tcc*[r]{movies with features}
    $filter\_by\_votes(\mathbb{M}, minimum\_votes)$\;
    $sort(\mathbb{M}, release\_date)$\; 
    \ForEach{m $\in$ $\mathbb{M}$}{
        $\mathbb{G}.update(\mathbb{P}m)$  \tcc*[r]{add movie producers}
        \If{$any(\mathbb{P}m \cap \mathbb{G}.giant\_component)$}{
            x $\leftarrow$ $m$.success\_parameters\;
            f $\leftarrow$ calculate\_features$(m)$\;
            movie\_map.add$(m, x, f)$\;
        }
    }
    normalize\_features$(movie\_map)$\;
    add\_product\_of\_features$(movie\_map)$\;
    models $\leftarrow$ [~] 		\tcc*[r]{regression models}
    \For(\tcc*[f]{30 runs}){$i\leftarrow 1$ \KwTo{} $30$}{
        R2,model $\leftarrow$ ramdom\_cross\_validation$(movie\_map)$;
        models.add$([R2, model])$;
    }
    return \{models.mean, models.confidenceInterval\}\;
\caption{Movie Prediction Task}\label{algo:predictor}
\end{algorithm}

\begin{table}[!ht]
\centering
\caption{\label{tab:test_sets}Three test sets for evaluating prediction accuracy.}
\begin{tabular}{@{}lcc@{}}
\toprule
Year Range & Number of Movies & Proportion of complete set \\
\midrule
2008--2013     &~3,317 &~27\% \\
1995--2013     &~9,775 &~52\% \\
1930--2013     &12,250 &100\% \\
\bottomrule
\end{tabular}
\end{table}


\subsection{Predictive Evaluation}
\label{sub:predictive_evaluation}
The methodology for evaluating the prediction model consists in 
splitting the movies in three chronological groups: from 2008 to
2013 (3,317 movies = 27\%), from 1995 to 2013 (9,775 movies = 52\%), and from
1930 to 2013 (12,250 movies = 100\%). We use each group for
training and testing the regression models for each of the three success
parameters. Table~\ref{tab:test_sets} summarizes the three different experiment
sets.

To evaluate how the features that depend on team social characteristics add
predictive power to the model, experimental trials are performed with the same
set of movies $\mathbb{M}$, but using different sets of features. Specifically,
we evaluate the regression model three times by considering: only the
non-topological features, only the topological features and all features.
Finally, we use the coefficient of determination~$R^2$ to assess how well the
regression models are predicting movie success parameters.
