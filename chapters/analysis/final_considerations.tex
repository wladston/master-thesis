\subsection{Final Considerations}
\label{sec:an:final}
In this chapter, we have confirmed that it is indeed possible to predict
success by considering only data available \textit{before} the movie is
released. Also, considering a large dataset for the prediction task is good;
however, spanning it through many years may negatively affect the quality of
the regression models. Finally, for all cases, the prediction model with
topological features combined with past success and experience have
outperformed the others, confirming this work main hypothesis.

Besides identifying the most relevant features, it is also important to assess
their \textit{strength} by evaluating their coefficients in the regression
model. Note that features with a higher coefficient have more impact in the
predicted value. Hence, we take an already trained regression model (the one
for votes, whose fitting was the best among the three parameters) and sort its
coefficients in descending magnitude. The coefficients with the highest
magnitudes are: the team's mean previous gross, mean previous experience, the
harmonic mean of clustering coefficients, contracted degree, genres Drama and
Documentaries, runtime and continent Africa. 

Out of all topological features, the impact of the \textit{clustering
coefficient} and \textit{degree} are significantly higher than the others
(their actual values are informed in the illustrative samples on the hits and
misses analyzes in Section~\ref{sec:hitsmisses}). Also, teams with more ties to
producers outside the current team are more likely to succeed, as are teams
with the lowest levels of clustering coefficients. These indicate that more
homogeneous teams tend to perform worse. In other words, this is a very
important result and suggests that assembling heterogeneous teams with many
\textit{external connections} is key in forming a team with higher success
odds. It is important to notice that such results reinforce the theory of weak
ties (or structural holes) that determine the importance of having nodes acting
as \textit{bridges} within a successful
network~\citep{Burt04,burt2005brokerage,newman2001structure}.
