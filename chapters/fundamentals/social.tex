\section{Social Network Analysis}
\label{sec:fundamentals:social}

\textit{Social Network Analysis} (SNA) emerged in the 20th century through the
combination of network and graph theories as a fundamental approach to
understanding social structures. SNA inherits from \textit{graph theory} the
description of these structures as sets of \textit{nodes} connected by
\textit{ties} (or \textit{edges}). SNA typically involves different metrics to
evaluate \textit{connections}, \textit{distributions} and \textit{segmentation}
of a given network.

Metrics can be calculated from a node in the network, or from the network as a
whole, to quantitatively characterize network aspects. For instance, centrality
metrics such as closeness (introduced by~\cite{bavelas}) and betweenness
(introduced by~\cite{anthonisse1971rush}) could capture the importance or
relevance of the node in the graph. Social metrics providing information that
is related only to a single node are referred to as \textit{ego-metrics}. Node
clustering metrics \citep{Holland01051971} and network constraints
\citep{Burt04} could indicate how heterogeneous the pool of connections of the
node is and thus its access to new information. Node degree (i.e., the number
of ties connecting it to other nodes) indicates how large the network of a node
is, possibly indicating the amount of support the node receives in its
productions.

Besides these \textit{ego-metrics}, there are those to characterize the nature
of the relationship between a pair of nodes. For instance, the \textit{number
of shared neighbours} is the number of nodes to which both nodes in the pair
are connected, whereas the \textit{neighborhood overlap} is the ratio of the
number of shared friends to all nodes connected to the pair. These metrics
provide an indication of how socially close nodes in a pair are.

Finally, there are network metrics that refer to characteristics of the
networks as a whole. Important metrics of this type include the
\textit{clustering coefficient} (measures the degree to which nodes in the
network tend to cluster) and the \textit{average shortest path length} (the
average of the shortest path lengths between all possible pairs of nodes in the
graph). The \textit{small world coefficient} is the ratio of the global
clustering coefficient to the average shortest path length. This measure is
related to the connectivity and cohesion among elements in a network. The more
a network exhibits characteristics of a small world, the more connected the
agents are to each other and connected to agents who know each other through
past collaborations~\citep{uzzi2005collaboration}.

Networks represented in graphs can be one-mode, with all nodes in the network
of the same type, or two-mode, with two distinct types of nodes. Two-mode
networks in which no ties exist connecting nodes from the same type (i.e., all
ties connect two nodes from distinct types) are \textit{bipartite networks}.
For instance, a bipartite graph can represent an online bulletin board in
which nodes are either users or discussions topics, and ties exist only between
users and topics to indicate whether a user participated in a topic. A very
famous class of two-mode networks are \textit{collaboration networks}, in which
one type of node represents human made artifacts (books, research papers,
movies, Broadway musicals) and the other represents people who worked in
creating the artifact. The network studied here is a two-mode network in which
nodes are either movies or movie producers, and edges indicate that a movie
producer worked in a movie.

It is also possible to transform a two-mode network into a one-mode
network~\cite{newman2003structure}. In this method, one type of node is chosen
and a new network is created only with the chosen type of node. In the new
(projected) network, ties are created between nodes that are connected to a
common node in the original network. For collaboration networks, when
projecting the original network for the authors, the projected network
represents authors with ties linking authors that worked together.

Two-mode networks, such as collaboration networks, are rarely analyzed without
prior transformation into one-mode networks, because the network metrics are
not defined for or adapted to two-mode networks. However, projected networks
always lose information from the original network, and calculating social
metrics from projected networks requires special attention.

For instance, clustering coefficients from projected one-mode networks can be
numerically high only due to the two-mode nature inherent to the original
network. For accurately obtaining the network measures from a bipartite
network, a more complex analysis considers random graphs, as described by
\cite{newman2003structure}. Also, a new clustering technique, called squared
clustering, was developed by~\cite{lind2005cycles} to better identify the
clustering effect in networks projected from bipartite networks. These
techniques (comparison to analog random graphs and square clustering) are used
in the present work in order to extract social information more accurately.

Most measures in bipartite networks are for single nodes. However, we need
network measurements from \textit{teams}, which are groups of nodes. In these
cases, one possible approach is node contraction: a group of nodes is replaced
by a newly created single node~\citep{gross2005graph, west2001introduction}.
Edges reaching the initial nodes then reach the new node. If there exist
multiple edges from the same origin reaching the new node, they are combined
into a new single edge. In this case, the new edge has the sum of the weights
of the previous edges. Extra information (such as past success and experience)
carried by the individual nodes and edges which are now combined can be joined
into the created node or edge. Finally, all the aforementioned ego-based
metrics can be calculated on such a \textit{supernode}. Note that the node
contraction technique cannot be used for aggregating pair-wise metrics among
all pairs in a group of nodes, since it aggregates all information into a
single node.
