\section{Team Success and Network Topology}
\label{sec:closely}
The main focus here is to use topological network characteristics from teams to
improve team success forecasting. Hence, we now go over approaches that look
for relations between performance and social characteristics.

\cite{chen2010impact} study the network formed by the collaboration among 16
countries and show the small world coefficient of the whole network is
correlated to more patent registrations from those countries.
\cite{schilling2007interfirm} study the collaboration among 1,106 companies and
find the small world metric is correlated to knowledge creation inside
companies and innovation. Similarly,~\cite{nemoto2011social} show the more
Wikipedia\footnote{Wikipedia:~\url{http://www.wikipedia.org}} editors with
higher social capital (taking part in a cohesive and centralized cluster)
working in a wikipedia article, the faster the article reaches higher quality
classifications.~\cite{singh2011network} find specific kinds of network ties
among open source developers are correlated with the development of more
popular open source projects.

In a large study of Flicker,~\cite{PapagelisMZ11} show nodes with higher degree
have more influence than nodes with lesser degree, regardless of the node's
credibility rankings. In the context of scientific research,~\cite{LiLY2013}
show social capital can be estimated with metrics from social networks
analysis, and researchers with more social capital publish research that has
more impact and reachability.

Higher performance is often related to social capital deriving from increased
access to Structural Holes, which in turn come from a privileged position
occupied by the node within the social network~\cite{Burt04,burt2005brokerage}.
Burt claims competitive advantage is mainly derived from this phenomena and
provides several ways in which individuals and business can occupy positions of
less network constraint~\citep{burt2009structural}. Indeed,~\cite{Rebehy2013}
investigated the interaction among 17 real state agents working on the same
office and found that agents with higher network constraints had significantly
worse performances in comparison to other agents.

Similar to us,~\cite{uzzi2005collaboration} evaluated the network of Broadway
musical producers (choreographers, writers and directors, not the cast) and
found the artistic and financial success of such a network as a whole is
correlated to its small world coefficient. The authors analyzed many network
metrics and found some of those were correlated to success. Nonetheless, all
these studies do not explore multivariate analysis of many network metrics.
Also, when considering the aggregation of network metrics, previous research
only employs the simple mean value among the agents' metrics, which can lead to
bias because drastically different distributions can have the same mean value.

Regarding the prediction of movie success,~\cite{KimKHC13} propose to mine
public opinion and trends in order to predict it. However, their experimental
dataset considers only 200 chosen movies (i.e., with clearly distinct features
that may produce biased results) and their analysis is limited to the
classification in four different success bins. Likewise,~\cite{OghinaBTR2012}
propose to predict IMDb movie ratings based on textual comments from social
media (Twitter and YouTube). They also consider a limited set of 70 movies for
experimental evaluation.

\cite{Ghiassi2015} present a classifier that can predict movie's gross in
terms of nine distinct classes.  However, they consider a dataset of only 364
movies (1999--2010), which contains abstract information such as the value the
movie production has from its famous actors, or whether the special effects and
the movie's competition are ``High'', ``Medium'' or ``Low''. They also include
data purchased from a media research company that is hard to obtain for many
movies (e.g., pre-release marketing expenses).
