\section{General Research on Team Performance Analysis and Social Network
Analysis}
\label{sec:loosely}
\cite{newman2001structure} was one of the first to explore metrics from large
collaboration networks. He demonstrated how to construct a social network from
bibliographic data and how to extract interesting characteristics from the
social graph by using complex networks metrics. He showed that different
scientific communities form small-world networks and are highly clustered, and
proposed a method for estimating tie strength. The technique used in this
thesis to extract a social network from collaboration data closely follows
Newman's work.

After the foundations for analytically extracting characteristics from social
networks, and with the wider availability of social network data and processing
power, this research area expanded across many academic disciplines, inspiring
innovative work.~\cite{lutter2013there} used the IMDb dataset and social
network analysis to explore gender inequality by tracking differences between
social features from actors and actresses. He found some network metrics are
link to better chances of career advancement, and that women are more likely to
drop out of their careers in comparison to men.

In another interesting application of social network
analysis,~\cite{wasserman2015cross} also used IMDb movie data for estimating
the relevance of a movie according to the network of cross references movies
received. They considered the inclusion in the US Library of Congress National
Film Registry as the ground truth for higher significance, and the
contributions are more accurate in predicting whether a movie is highly
relevant than the combined opinion of movie experts. On the other hand, we
examine movie success in a broader context that includes a combination of
popularity, financial success and public acceptance.

Studies in different social contexts try to understand how people work together
to better achieve their goals.~\cite{stokols2008} explore the ``Science of team
science'', focusing on the processes by which scientific teams organize and
conduct their work. Specifically, they explore how teams connect and
collaborate in order to achieve breakthroughs that would not be attainable by
either individual or simple additive efforts.

Likewise, others identify correlations between social characteristics and team
performance.~\cite{grund2012network} performed social network analysis in
football teams and found that network characteristics in the graph that models
the way players pass the ball is related to team winning odds. Unlike those
work, here we focus on the social characteristics of teams in relation to their
connections external to the team, instead of team's internal links.

Although there is a vast pool of previous work connecting social structure and
productivity, few of them investigate social features in the context of the
team formation problem. Moreover, the team formation problem is well
established and concerns strategies and algorithms for efficiently forming
teams in the best possible way. Many of previous work on this problem focus on
efficiently selecting elements with different skill sets in order to form
multi-functional teams.

For example,~\cite{anagnostopoulos2012online} present a technique for
efficiently forming teams with a minimum set of combined skills. They also use
the IMDb dataset to gather sample data to an emulated problem in which teams
need to be formed from directors with a minimum combined skill set. Also, the
genres of movies they have previously acted serve as the skill.

\cite{tseng2004novel} also provide techniques for forming teams with a best
possible multiple skillset.~\cite{wi2009team} form teams to work on given
projects by matching nodes that have the specific knowledge related to the
keywords present in the project's description.  Deterministic team formation
extends even to autonomous robots:~\cite{GunnA15} present techniques for
autonomous robots (working in disaster zones) to dynamically form teams to
better perform the task of locating humans in need of rescue.

Differently from these works, we contribute to the team formation problem not
by efficiently finding elements to assemble a team that must satisfy a given
restriction, but by identifying \textit{social patterns} present in teams that
perform better in the context of producing successful movies. Moreover, other
approaches on team formation tackle it as an optimization problem by grouping
elements in order to most harmonically distribute the different abilities
across multifunctional teams~\citep{anagnostopoulos2012online}. Other studies
explore the presence of elements with special characteristics in the team to
understand its success. For instance,~\cite{elberse2007power} studies the
impact of extremely famous actors in movie teams. We complement such prior
studies by focusing in accessing the effect of the social arrangement of the
agents in the network, rather than the effect of individual characteristics of
team members.
